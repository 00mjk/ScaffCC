\chapter{Installation}\label{ch:inst}

\section{Getting ScaffCC}
\begin{enumerate}
  \item Go to https://github.com/epiqc/ScaffCC
  \item Download the repository:
\begin{lstlisting}
   git clone https://github.com/epiqc/ScaffCC.git [dir]
\end{lstlisting}
\end{enumerate}


\section{Building ScaffCC}
First you need to install the following dependencies. For each one, you can either install by building from source,
or use the package manager of your system (``yum" on Red Hat or ``apt-get" on Ubuntu).  

\begin{enumerate}
  \item Static libraries for libstdc++ and glibc

  \begin{itemize}
    \item ``Ubuntu" \\
    Install GNU gold linker \\
    You can check if you have this now by doing `ld -v' and if it says `GNU gold' you have it
    \begin{lstlisting}
    sudo apt-get install binutils-gold
    \end{lstlisting}

    \item ``Red Hat"
    \begin{lstlisting}
    sudo yum install libstdc++-static 
    sudo yum install glibc-static
    \end{lstlisting}
  \end{itemize}
  
  \item GCC 4.5 or higher
  NOTE: if you need to preserve an older build, consider using `update-alternatives' as system-wide method for preserving and maintaining these.

  \item Boost 1.48
  \begin{itemize}
    \item ``Source Build" \\
    Boost installation instructions are here: \\
    \url{http://www.boost.org/doc/libs/1_48_0/doc/html/bbv2/installation.html}
    \begin{lstlisting}
    wget http://sourceforge.net/projects/boost/files/boost/1.48.0/boost_1_48_0.tar.gz
    tar zxf boost_1_48_0.tar.gz && cd boost_1_48_0/
    sudo ./bootstrap.sh
    sudo ./b2 install --prefix=/usr/local
    \end{lstlisting}
    Alternatively, Ubuntu users can install the current Boost version via:
    \begin{lstlisting}
    sudo apt-get install libboost-all-dev
    \end{lstlisting}
  \end{itemize}

  \item The GNU Multiple Precision Arithmetic Library (gmp and gmpxx)    
  \begin{itemize}
    \item ``Ubuntu"
    Use tab-completion to verify the correct packages (e.g. libgmp\textless tab\textgreater \textless tab\textgreater)
    \begin{lstlisting}
    sudo apt-get install libgmp-dev libgmpxx4ldbl
    \end{lstlisting}

    \item ``Source Build"
    \begin{lstlisting}
    wget https://ftp.gnu.org/gnu/gmp/gmp-6.0.0a.tar.bz2
    sudo ./configure --disable-shared --enable-static --enable-cxx
    sudo make && sudo make check && sudo make install
    \end{lstlisting}
  \end{itemize}

  \item The GNU MPFR Library (mpfr)
  \begin{itemize}    
    \item ``Ubuntu"
    \begin{lstlisting}
    sudo apt-get install libmpfr-dev
    \end{lstlisting}

    \item ``Source Build"
    \begin{lstlisting}
    wget http://www.mpfr.org/mpfr-current/mpfr-3.1.4.tar.bz2
    sudo ./configure --disable-shared --enable-static
    sudo make && sudo make check && sudo make install
    \end{lstlisting}
  \end{itemize}

  \item Python 2.7 (or later)

  \item CMake (For Integrating RKQC Functionality)
  \begin{itemize}
    \item ``Ubuntu"
    \begin{lstlisting}
    sudo apt-get install cmake
    \end{lstlisting}

    \item ``Source Build" \\
    There are instructions for downloading and building CMake from source at: \\
    \url{https://cmake.org/install}
  \end{itemize}

\end{enumerate}

Once you have all of the required libraries, simply run 
\begin{lstlisting}
make
\end{lstlisting} 
or 
\begin{lstlisting}
make USE_GCC=1
\end{lstlisting} 
at the root of the repository. The $USE\_GCC$ flag will force the 
Makefile to use GCC to compile instead, and this has been seen to be faster 
on some systems.

\section{Verifying Installation}

Included with ScaffCC is a suite of tests designed to verify the integrity of new installations. To access these tests and verify that the installation process completed successfully:

\begin{enumerate}
    \item Enter the scripts/ directory:
    \begin{lstlisting}
        cd scripts/
    \end{lstlisting}
    \item Run the Regression Test Suite:
    \begin{lstlisting}
        ./regression_test.sh
    \end{lstlisting}
\end{enumerate}

This invokation will compile three small benchmarks, and verify that the generated files match those precompiled on an existing system, which are included in the test cases directory. If the three tests complete with a ``Succeeded", the installation was successful.
