\section{Declaration of Circuits}

The construction of RKQC is designed to be as similar as possible to C++ programs which contain a number of functions. To this end, circuits are specifiable through a main function, as well as through various submodules that are declared exactly as common C++ functions. Exactly one module must be named "int main", however this module, as well as any module, can call any other module at any time.

\section{Signal Types}

There is one major type of signal in RKQC, the \textbf{qint}. This type forms the basis for the five other types of signals that are derived from this main class:

\begin{itemize}
    \item qbit
    \item zero\_to\_garbage ancilla
    \item zero\_to\_zero ancilla
    \item one\_to\_garbage ancilla
    \item one\_to\_one ancilla
\end{itemize}

All of the above signals can be declared and used in the specification of gates and circuit descriptions. The primary differences are that the "\_to\_garbage" ancilla are ancilla signals that are not guaranteed to maintain their original state during computation, while the two other ancilla types do make this guarantee.

As will be explained later on, all of these signals are of a base type \textbf{qint}, which allows for circuit description flexibility when creating and passing parameters to submodules.

\section{Creating and Using Modules}

At the circuit level, both user defined modules as well as built in modules can be instantiated in the same fashion. 

In the example circuit "e008\_swap.cpp", the built in function "cnot" is called, creating CNOT gates on the qubits passed in as parameters:

\begin{lstlisting}
    int main( int argc, char ** argv )
    {
        qbit a;
        qbit b;

        cnot(a, b);
        cnot(b, a);
        cnot(a, b);

        return 0;
    }
\end{lstlisting}

In exactly the same fashion, a circuit designer can create a submodule, and instantiate it within the main function, as in file "e009\_swap\_triples.cpp":

\begin{lstlisting}
    void swap_bits( qint x, qint y){
        cnot(x, y);
        cnot(x, y);
        cnot(x, y);
    }

    int main( int argc, char ** argv ){
        qbit a0;
        qbit a1;
        qbit a2;

        qbit b0;
        qbit b1;
        qbit b2;

        swap_bits(a0, b0);
        swap_bits(a1, b1);
        swap_bits(a2, b2);

        return 0;
    }
\end{lstlisting}

\section{Arrays}

RKQC also allows for multi-dimensional arrays of signals to be declared and used in circuits. To demonstrate this, file "e010\_swap\_triples\_2.cpp" replicates the functionality of "e009\_swap\_triples.cpp" with arrays:

\begin{lstlisting}
    void swap_bits( qint x, qint y){
        cnot(x, y);
        cnot(x, y);
        cnot(x, y);
    }

    int main( int argc, char ** argv ){
        qbit a(3);
        qbit b(3);

        swap_bits( a[0], b[0] );
        swap_bits( a[1], b[1] );
        swap_bits( a[2], b[2] );

        return 0;
    }
\end{lstlisting}

\section{Signals and Gates}

Gates are built in modules that can be instantiated in a variety of methods, with a variety of signal types. Specifically, 
\begin{itemize}
    \item single qint signals,
    \item qint arrays,
    \item single qbit signals,
    \item qbit arrays,
    \item single ancilla signals, and
    \item ancilla arrays
\end{itemize}
all are capable of being passed through as parameters to built in gates. For an example of this, file "e002\_signals\_and\_gates.cpp" calls all of the intrinsic built in gates with all of the different signal types: 
\begin{lstlisting}
    int main( int argc, char ** argv )
    {
    
        qbit a;
        qbit b;
        qbit c;
        qbit d(10);
    	qbit e(10);
    	qbit f(10);
    /*------------ Gates Called By Qubit --------------*/
        NOT(a);
        cnot(a,b);
        toffoli(a, b, c);	
    /*------------ Gates Called By Register Index -----*/ 
        NOT(d[0]);
        cnot(d[1], d[9]);
        toffoli(d[0], e[0], c[0]);
    /*------------ Gates Called By Mixed Register Index & Qbits -----*/ 
        cnot(a, d[0]);
        cnot(e[0], b);
    	toffoli(d[0], a, b);
    	toffoli(a, d[0], b);
    	toffoli(a, b, d[0]); 
    	toffoli(d[0], e[0], a);
    	toffoli(d[0], a, e[0]);
    	toffoli(a, d[0], e[0]);
    /*------------ Gates Called By Full Qbit Reg ------*/ 
    	NOT(d);
    	cnot(d, e);
    	toffoli(d, e, f);
    
        return 0;
    }
\end{lstlisting}

\section{Assign Value}

A commonly used function of assigning value of one signal to another is built into RKQC in the form of a function:
\begin{lstlisting}
    assign_value_of_b_to_a();
\end{lstlisting}

This function can be instantiated a number of different ways, with integer constants or with two \textbf{qint} signals. These instantiation methods are detailed in file "e004\_assign\_value.cpp":
\begin{lstlisting}
    int main( int argc, char ** argv )
    {
        qbit a;
        assign_value_of_b_to_a(a, "0", 1);
    
        qbit b(8);
        assign_value_of_b_to_a(b, "1", 8);
    
        qbit c(8);
        assign_value_of_b_to_a(c, b, 8);
    
        return 0;
    }
\end{lstlisting}

\section{Integer Arithmetic Modules}

RKQC comes with implementations of quantum adders and multipliers developed in recent theoretical work. For instance, the quantum ripple-carry adder implemented in the function: 

\begin{lstlisting}
    a_eq_a_plus_b();
\end{lstlisting}

is derived from work by Wang et. al \cite{adder}, and an example of usage is contained in file "e005\_adder.cpp": 

\begin{lstlisting}
    int main( int argc, char ** argv )
    {
        qbit a(32);
        qbit b(32);
    
        a_eq_a_plus_b(a,b,32);
    
        return 0;
    }
\end{lstlisting}

Alternatively, an implementation of an ancilla-free addition circuit is also contained within RKQC, derived from earlier work by Cuccaro \cite{Cuccaro}. 
